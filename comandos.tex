\newtheoremstyle{estiloejemplo}%
    {9pt}{9pt}%
    {}%
    {0pt}%
    {\bfseries\scshape}{.}%
    { }%
    {}

\theoremstyle{estiloejemplo}
    \newtheorem{ejemplo}{Ejemplo}
    \newtheorem*{observacion}{Observaci�n}

\newtheoremstyle{estiloteorema}%
    {9pt}{9pt}%
    {\slshape}%
    {0pt}%
    {\bfseries\scshape}{.}%
    { }%
    {}

\theoremstyle{estiloteorema}
    \newtheorem{definicion}{Definici�n}[chapter]
    \newtheorem{axioma}{Axioma}
    \newtheorem{teorema}{Teorema}[chapter]
    \newtheorem{corolario}[teorema]{Corolario}
    \newtheorem{proposicion}[teorema]{Proposici�n}
    \newtheorem{lema}[teorema]{Lema}

\renewcommand{\emph}{\textit}

\newcommand{\funcion}[5]{%
{\setlength{\arraycolsep}{2pt}
\begin{array}{r@{}ccl}
#1\colon & #2 & \longrightarrow & #3\\
& #4 & \longmapsto & #5
\end{array}}}

\newcommand{\func}[3]{%
#1\colon  #2  \to  #3
}

\newcommand{\cl}[1]{\overline{#1}}

% --- Conjuntos
\newcommand{\reales}{\mathbb{R}}
\newcommand{\R}{\mathbb{R}}
\newcommand{\naturales}{\mathbb{N}}
\newcommand{\N}{\omega}
\newcommand{\K}{\mathcal{K}}
\newcommand{\Kp}{\mathscr{K}}
\renewcommand{\P}{\mathcal{P}}
\newcommand{\OR}{\bm{OR}}

%\renewcommand{\baselinestretch}{1.5}

% --- Sucesiones
\newcommand{\suc}[2][n]{\left(#2\right)_{#1\in\naturales}}
\newcommand{\ssuc}[2][n]{(#2)_{#1\in\naturales}}

% --- Comentario
\newcommand{\comentario}[1]{{\color{red!50!black}\textsf{#1}}}
\newenvironment{Comentario}{
    \bgroup
    \color{red!50!black}
    \sffamily
    }
    {
    \egroup
    }

\setlist[itemize,1]{label=$\bullet$}

\allowdisplaybreaks 